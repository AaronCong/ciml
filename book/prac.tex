\chapter{Practical Issues} \label{sec:prac}

%start at 3 end at 14; left is -0.8 right is 15.4
%\begin{textblock}{1}(15.4,\chapterlabel)\tikz\node [fill=darkergreen,text=white,draw,circle] (chnode) {\normalfont\sffamily\Huge\bfseries\scshape \chapterlabel~~~};\end{textblock}
%\begin{textblock}{1}(-0.8,14)\tikz\node [fill=darkergreen,text=white,draw,circle] (chnode) {\normalfont\sffamily\Huge\bfseries\scshape ~~~\chapterlabel};\end{textblock}

\chapterquote{In theory, there is no difference between theory and practice.\\But, in practice, there is.}{Jan~L.A.~van~de~Snepscheut}

\begin{learningobjectives}
\item Translate between a problem description and a concrete learning
  problem.
\item Perform basic feature engineering on image and text data.
\item Explain how to use cross-validation to tune hyperparameters and
  estimate future performance.
\item Compare and contrast the differences between several evaluation
  metrics.
\item Explain why feature combinations are important for learning with
  some models but not others.
\item Explain the relationship between the three learning techniques
  you have seen so far.
\item Apply several debugging techniques to learning algorithms.
\end{learningobjectives}

TODO: one to two examples per feature

\dependencies{\chref{sec:dt},\chref{sec:knn},\chref{sec:perc}}

\newthought{At this point,} you have seen three qualitatively
different models for learning: decision trees, nearest neighbors, and
perceptrons.  You have also learned about clustering with the
$K$-means algorithm.  You will shortly learn about more complex
models, most of which are variants on things you already know.
However, before attempting to understand more complex models of
learning, it is important to have a firm grasp on how to use machine
learning in practice.  This chapter is all about how to go from an
abstract learning problem to a concrete implementation.  You will see
some examples of ``best practices'' along with justifications of these
practices.

In many ways, going from an abstract problem to a concrete learning
task is more of an art than a science.  However, this art can have a
huge impact on the practical performance of learning systems.  In many
cases, moving to a more complicated learning algorithm will gain you a
few percent improvement.  Going to a better representation will gain
you an order of magnitude improvement.  To this end, we will discuss
several high level ideas to help you develop a better artistic
sensibility.

\section{The Importance of Good Features}

Machine learning is magical.  You give it data and it manages to
classify that data.  For many, it can actually outperform a human!
But, like so many problems in the world, there is a significant
``garbage in, garbage out'' aspect to machine learning.  If the data
you give it is trash, the learning algorithm is unlikely to be able to
overcome it.

Consider a problem of object recognition from images.  If you start
with a $100 \times 100$ pixel image, a very easy feature
representation of this image is as a $30,000$ dimensional vector,
where each dimension corresponds to the red, green or blue component
of some pixel in the image.  So perhaps feature 1 is the amount of red
in pixel $(1,1)$; feature 2 is the amount of green in that pixel; and
so on.  This is the \concept{pixel representation} of images.

\TODOFigure{prac:imagepix}{object recognition in pixels}

One thing to keep in mind is that the pixel representation
\emph{throws away} all locality information in the image.  Learning
algorithms don't care about features: they only care about feature
values.  So I can \emph{permute} all of the features, with no effect
on the learning algorithm (so long as I apply the same permutation to
all training and test examples).  Figure~\ref{fig:prac:imagepix} shows
some images whose pixels have been randomly permuted (in this case only
the pixels are permuted, not the colors).  All of these objects are
things that you've seen plenty of examples of; can you identify them?
Should you expect a machine to be able to?

\TODOFigure{prac:imagepatch}{object recognition in patches}
\TODOFigure{prac:imageshape}{object recognition in shapes}

An alternative representation of images is the \concept{patch
  representation}, where the unit of interest is a small rectangular
block of an image, rather than a single pixel.  Again, permuting the
patches has no effect on the classifier.
Figure~\ref{fig:prac:imagepatch} shows the same images in patch
representation.  Can you identify them?  A final representation is a
\concept{shape representation}.  Here, we throw out all color and
pixel information and simply provide a bounding polygon.
Figure~\ref{fig:prac:imageshape} shows the same images in this
representation.  Is this now enough to identify them?  (If not, you
can find the answers at the end of this chapter.)

\TODOFigure{prac:bow}{BOW repr of one positive and one negative review}

In the context of \concept{text categorization} (for instance, the
sentiment recognition task), one standard representation is the
\concept{bag of words} representation.  Here, we have one feature for
each unique word that appears in a document.  For the feature
\feat{happy}, the feature value is the number of times that the word
``happy'' appears in the document.  The bag of words (BOW)
representation throws away all position information.
Figure~\ref{fig:prac:bow} shows a BOW representation for two
documents: one positive and one negative.  Can you tell which is
which?

\section{Irrelevant and Redundant Features}

One big difference between learning models is how robust they are to
the addition of noisy or irrelevant features.  Intuitively, an
irrelevant feature is one that is completely uncorrelated with the
prediction task.  A feature $f$ whose expectation does not depend on
the label $\Ep[f \| Y] = \Ep[f]$ might be irrelevant.  For instance,
the presence of the word ``the'' might be largely irrelevant for
predicting whether a course review is positive or negative.  

\thinkaboutit{Is it possible to have a feature $f$ whose expectation
  does not depend on the label, but is nevertheless still useful for
  prediction?}

A secondary issue is how well these algorithms deal with
\concept{redundant features}.  Two features are redundant if they are
highly correlated, regardless of whether they are correlated with the
task or not.  For example, having a bright red pixel in an image at
position $(20,93)$ is probably highly redundant with having a bright
red pixel at position $(21,93)$.  Both might be useful (e.g., for
identifying fire hydrants), but because of how images are structured,
these two features are likely to co-occur frequently.

When thinking about robustness to irrelevant or redundant features, it
is usually not worthwhile thinking of the case where one has $999$
great features and $1$ bad feature.  The interesting case is when the
bad features outnumber the good features, and often outnumber by a
large degree.  For instance, perhaps the number of good features is
something like $\log D$ out of a set of $D$ total features.  The
question is how robust are algorithms in this case.\sidenote{You might
  think it's crazy to have so many irrelevant features, but the cases
  you've seen so far (bag of words, bag of pixels) are both reasonable
  examples of this!  How many words, out of the entire English
  vocabulary (roughly $10,000-100,000$ words), are actually useful for
  predicting positive and negative course reviews?}

For shallow \concept{decision trees}, the model explicitly selects
features that are highly correlated with the label.  In particular, by
limiting the depth of the decision tree, one can at least \emph{hope}
that the model will be able to throw away irrelevant features.
Redundant features are almost certainly thrown out: once you select
one feature, the second feature now looks mostly useless.  The only
possible issue with irrelevant features is that even though they're
irrelevant, they \emph{happen to} correlate with the class label on
the training data, but chance.

As a thought experiment, suppose that we have $N$ training examples,
and exactly half are positive examples and half are negative
examples.  Suppose there's some binary feature, $f$, that is completely
uncorrelated with the label.  This feature has a 50/50 chance of
appearing in any example, regardless of the label.  In principle, the
decision tree should \emph{not} select this feature.  But, by chance,
especially if $N$ is small, the feature might \emph{look} correlated
with the label.  This is analogous to flipping two coins
simultaneously $N$ times.  Even though the coins are independent, it's
entirely possible that you will observe a sequence like $(H,H), (T,T),
(H,H), (H,H)$, which makes them look entirely correlated!  The hope is
that as $N$ grows, this becomes less and less likely.  In fact, we can
explicitly compute how likely this is to happen.

To do this, let's fix the sequence of $N$ labels.  We now flip a coin
$N$ times and consider how likely it is that it exactly matches the
label.  This is easy: the probability is $0.5^N$.  Now, we would also
be confused if it exactly matched \emph{not} the label, which has the
same probability.  So the chance that it looks perfectly correlated is
$0.5^N + 0.5^N = 0.5^{N-1}$.  Thankfully, this shrinks down very small
(e.g., $10^{-6}$) after only $21$ data points.

This makes us happy.  The problem is that we don't have one irrelevant
feature: we have $D-\log D$ irrelevant features!  If we randomly pick
two irrelevant feature values, each has the same probability of
perfectly correlating: $0.5^{N-1}$.  But since there are two and
they're independent coins, the chance that \emph{either} correlates
perfectly is $2 \times 0.5^{N-1} = 0.5^{N-2}$.  In general, if we have
$K$ irrelevant features, all of which are random independent coins,
the chance that at least one of them perfectly correlates is
$0.5^{N-K}$.  This suggests that if we have a sizeable number $K$ of
irrelevant features, we'd better have at \emph{least} $K+21$ training
examples.

Unfortunately, the situation is actually worse than this.  In the
above analysis we only considered the case of \emph{perfect}
correlation.  We could also consider the case of \emph{partial}
correlation, which would yield even higher probabilities.  (This is
left as Exercise~\ref{ex:prac:dtbound} for those who want some
practice with probabilistic analysis.)  Suffice it to say that even
decision trees can become confused.

\TODOFigure{prac:addirrel}{data from high dimensional warning, interpolated}

In the case of \concept{$K$-nearest neighbors}, the situation is
perhaps more dire.  Since KNN weighs each feature just as much as
another feature, the introduction of irrelevant features can
completely mess up KNN prediction.  In fact, as you saw, in high
dimensional space, randomly distributed points all look approximately
the same distance apart.  If we add lots and lots of randomly
distributed features to a data set, then all distances still
converge.  This is shown experimentally in
Figure~\ref{fig:prac:addirrel}, where we start with the digit
categorization data and continually add irrelevant, uniformly
distributed features, and generate a histogram of distances.
Eventually, all distances converge.

In the case of the \concept{perceptron}, one can \emph{hope} that it
might learn to assign zero weight to irrelevant features.  For
instance, consider a binary feature is randomly one or zero
independent of the label.  If the perceptron makes just as many
updates for positive examples as for negative examples, there is a
reasonable chance this feature weight will be zero.  At the very
least, it should be small.

\thinkaboutit{What happens with the perceptron with truly redundant
  features (i.e., one is literally a copy of the other)?}

\TODOFigure{prac:noisy}{dt,knn,perc on increasing amounts of noise}

To get a better practical sense of how sensitive these algorithms are
to irrelevant features, Figure~\ref{fig:prac:noisy} shows the
\emph{test} performance of the three algorithms with an increasing
number of compltely noisy features.  In all cases, the hyperparameters
were tuned on validation data.  TODO...

\section{Feature Pruning and Normalization}

In text categorization problems, some words simply do not appear very
often.  Perhaps the word ``groovy''\sidenote{This is typically
  positive indicator, or at least it was back in the US in the 1970s.}
appears in exactly one training document, which is positive.  Is it
really worth keeping this word around as a feature?  It's a dangerous
endeavor because it's hard to tell with just one training example if
it is really correlated with the positive class, or is it just noise.
You could hope that your learning algorithm is smart enough to figure
it out.  Or you could just remove it.  That means that (a) the
learning algorithm won't have to figure it out, and (b) you've reduced
the number of dimensions you have, so things should be more efficient,
and less ``scary.''

\TODOFigure{prac:pruning}{effect of pruning on text data}

This idea of feature pruning is very useful and applied in many
applications.  It is easiest in the case of binary features.  If a
binary feature only appears some small number $K$ times (in the
training data: no fair looking at the test data!), you simply remove
it from consideration.  (You might also want to remove features that
appear in all-but-$K$ many documents, for instance the word ``the''
appears in pretty much every English document ever written.)  Typical
choices for $K$ are $1, 2, 5, 10, 20, 50$, mostly depending on the
size of the data.  On a text data set with $1000$ documents, a cutoff
of $5$ is probably reasonable.  On a text data set the size of the
web, a cut of $50$ or even $100$ or $200$ is probably
reasonable\sidenote{According to Google, the following words (among
  many others) appear $200$ times on the web: moudlings, agaggagctg,
  setgravity, rogov, prosomeric, spunlaid, piyushtwok, telelesson,
  nesmysl, brighnasa.  For comparison, the word ``the'' appears
  $19,401,194,714$ (19 billion) times.}.
Figure~\ref{fig:prac:pruning} shows the effect of pruning on a
sentiment analysis task.  In the beginning, pruning does not hurt (and
sometimes helps!) but eventually we prune away all the interesting
words and performance suffers.

\begin{mathreview}{Data Statistics: Means and Variances}
  data mean, variance, moments, expectations, etc...
\end{mathreview}

\TODOFigure{prac:variance}{effect of pruning on vision}

In the case of real-valued features, the question is how to extend the
idea of ``does not occur much'' to real values.  A reasonable
definition is to look for features with \emph{low variance}.  In fact,
for binary features, ones that almost never appear or almost always
appear will also have low variance.  Figure~\ref{fig:prac:variance}
shows the result of pruning low-variance features on the digit
recognition task.  Again, at first pruning does not hurt (and
sometimes helps!) but eventually we have thrown out all the useful
features.

\thinkaboutit{Earlier we discussed the problem of \emph{scale} of
  features (e.g., millimeters versus centimeters).  Does this have an
  impact on variance-based feature pruning?}

Once you have pruned away irrelevant features, it is often useful to
\concept{normalize} the data so that it is consistent in some way.
There are two basic types of normalization: \concept{feature
  normalization} and \concept{example normalization}.  In feature
normalization, you go through each feature and adjust it the same way
across all examples.  In example normalization, each example is
adjusted individually.

\TODOFigure{prac:transform}{picture of centering, scaling by variance and scaling by absolute value}

The goal of both types of normalization is to make it \emph{easier}
for your learning algorithm to learn.  In feature normalization, there
are two standard things to do:
\begin{enumerate}
\item Centering: moving the entire data set so that it is centered
  around the origin.
\item Scaling: rescaling each feature so that one of the following
  holds:
\begin{enumerate}
\item Each feature has variance $1$ across the training data.
\item Each feature has maximum absolute value $1$ across the training
  data.
\end{enumerate}
\end{enumerate}
These transformations are shown geometrically in
Figure~\ref{fig:prac:transform}.  The goal of centering is to make
sure that no features are arbitrarily large.  The goal of scaling is
to make sure that all features have roughly the same scale (to avoid
the issue of centimeters versus millimeters).

\thinkaboutit{For the three models you know about (KNN, DT,
  Perceptron), which are most sensitive to centering?  Which are most
  sensitive to scaling?}

These computations are fairly straightforward.  Here, $x_{n,d}$ refers
to the $d$th feature of example $n$.  Since it is very rare to apply
scaling without previously applying centering, the expressions below
for scaling assume that the data is already centered.
\begin{align}
\text{Centering:} && x_{n,d} &\leftarrow x_{n,d} - \mu_d \\
\text{Variance Scaling:} && x_{n,d} &\leftarrow x_{n,d} / \si_d \\
\text{Absolute Scaling:} && x_{n,d} &\leftarrow x_{n,d} / r_d \\
&\textit{where:}
& \mu_d &= \frac 1 N \sum_n x_{n,d} \\
&& \si_d &= \sqrt{ \frac 1 N \sum_n (x_{n,d} - \mu_d)^2 } \\
&& r_d   &= \max_n \ab{x_{n,d}}
\end{align}


In practice, if the dynamic range of your features is already some
subset of $[-2,2]$ or $[-3,3]$, then it is probably not worth the
effort of centering and scaling.  (It's an effort because you have to
keep around your centering and scaling calculations so that you can
apply them to the test data as well!)  However, if some of your
features are orders of magnitude larger than others, it might be
helpful.  Remember that you might know best: if the difference in
scale is actually significant for your problem, then rescaling might
throw away useful information.

One thing to be wary of is centering binary data.  In many cases,
binary data is very \emph{sparse}: for a given example, only a few of
the features are ``on.''  For instance, out of a vocabulary of
$10,000$ or $100,000$ words, a given document probably only contains
about $100$.  From a storage and computation perspective, this is very
useful.  However, after centering, the data will no longer be sparse and
you will pay dearly with outrageously slow implementations.

\TODOFigure{prac:exnorm}{example of example normalization}

In \concept{example normalization}, you view examples one at a time.
The most standard normalization is to ensure that the length of each
example vector is one: namely, each example lies somewhere on the unit
hypersphere.  This is a simple transformation:
\begin{align}
\text{Example Normalization:} && \vx_{n} &\leftarrow \vx_{n} / \norm{\vx_n}
\end{align}
This transformation is depicted in Figure~\ref{fig:prac:exnorm}.

The main advantage to example normalization is that it makes
comparisons more straightforward across data sets.  If I hand you two
data sets that differ only in the norm of the feature vectors (i.e.,
one is just a scaled version of the other), it is difficult to compare
the learned models.  Example normalization makes this more
straightforward.  Moreover, as you saw in the perceptron convergence
proof, it is often just mathematically easier to assume normalized
data.

\section{Combinatorial Feature Explosion}
\label{sec:prac:combinatorial}

You learned in Chapter~\ref{sec:perc} that linear models (like the
perceptron) cannot solve the XOR problem.  You also learned that by
performing a combinatorial feature explosion, they could.  But that
came at the computational expense of gigantic feature vectors.

\begin{comment}
To make things more formal, suppose that you are given a feature
vector $\vx$ in $D$-dimensional space.  Call these original features
the ``base features.''  There are several types of feature
combinations you could add:
\begin{enumerate}
\item Products: for all $d,d'$, create a feature whose value is $x_d
  \times x_{d'}$.  In the case of binary features, this corresponds to
  an ``and'' operation.
\item Sums: for all $d,d'$, create a feature whose value is $x_d
  + x_{d'}$.  In the case of binary features, this roughly corresponds to
  an ``or'' operation.
\item Log-transform: for all $d$, create a feature whose value is
  $\log x_d$.  (Only makes sense for non-negative features.)
\item Power-transform: for all $d$, create a feature whose value is
  $x_d^k$ for some $k$.
\end{enumerate}
Call features that are generated in these ways ``meta features.''
Now, there is nothing preventing you from combining meta features into
more complex ``meta meta features'' and so on.  Eventually you could
have a feature like:
\begin{equation}
  x_5 \times \log x_2 + (\log x_{10} + x_{16})^5 \times \log (x_{1}^2
  + x_{5})
\end{equation}
Such a feature might not be pariticularly \emph{useful}, but you could
certainly construct it.

In principle any one of these ``meta meta features'' might be
helpful.  Unfortunately, there is a combinatorial explosion of
possibilities, even with just these four constructions (technically
``power'' is unnecessary since it can be obtained through repeated
products).  Moreover, as you know, learning algorithms can be fooled
by features that look good by chance, but aren't actually good.  By
generating this combinatorial explosion of features, you run the risk
of creating \emph{many} features that look good by chance.
\end{comment}

Of the algorithms that you've seen so far, the perceptron is the one
that has the most to gain by feature combination.  And the decision
tree is the one that has the least to gain.  In fact, the decision
tree construction is essentially building meta features for you.  (Or,
at least, it is building meta features constructed purely through
``logical ands.'')

\TODOFigure{prac:dttoperc}{turning a DT into a set of meta features}

This observation leads to a heuristic for constructing meta features
\emph{for} perceptrons \emph{from} decision trees.  The idea is to
train a decision tree on the training data.  From that decision tree,
you can extract meta features by looking at feature combinations along
branches.  You can then add \emph{only} those feature combinations as
meta features to the feature set for the perceptron.
Figure~\ref{fig:prac:dttoperc} shows a small decision tree and a set
of meta features that you might extract from it.  There is a
hyperparameter here of what length paths to extract from the tree: in
this case, only paths of length two are extracted.  For bigger trees,
or if you have more data, you might benefit from longer paths.

\TODOFigure{prac:log}{performance on text categ with word counts
  versus log word counts}

In addition to combinatorial transformations, the \concept{logarithmic
  transformation} can be quite useful in practice.  It seems like a
strange thing to be useful, since it doesn't seem to fundamentally
change the data.  However, since many learning algorithms operate by
linear operations on the features (both perceptron and KNN do this),
the log-transform is a way to get product-like operations.  The
question is which of the following feels more applicable to your data:
(1) every time this feature increases by one, I'm equally more likely
to predict a positive label; (2) every time this feature doubles, I'm
equally more like to predict a positive label.  In the first case, you
should stick with linear features and in the second case you should
switch to a log-transform.  This is an important transformation in
text data, where the presence of the word ``excellent'' once is a good
indicator of a positive review; seeing ``excellent'' twice is a better
indicator; but the difference between seeing ``excellent'' 10 times
and seeing it 11 times really isn't a big deal any more.  A
log-transform achieves this.  Experimentally, you can see the
difference in test performance between word count data and log-word
count data in Figure~\ref{fig:prac:log}.  Here, the transformation is
actually $x_d \mapsto \log_2(x_d + 1)$ to ensure that zeros remain
zero and sparsity is retained.

\section{Evaluating Model Performance}

So far, our focus has been on classifiers that achieve \emph{high
  accuracy}.  In some cases, this is not what you might want.  For
instance, if you are trying to predict whether a patient has cancer or
not, it might be better to err on one side (saying they have cancer
when they don't) than the other (because then they die).  Similarly,
letting a little spam slip through might be better than accidentally
blocking one email from your boss.

There are two major types of binary classification problems.  One is
``X versus Y.''  For instance, positive versus negative sentiment.
Another is ``X versus not-X.''  For instance, spam versus non-spam.
(The argument being that there are lots of types of non-spam.)  Or in
the context of web search, relevant document versus irrelevant
document.  This is a subtle and subjective decision.  But ``X versus
not-X'' problems often have more of the feel of ``X spotting'' rather
than a true distinction between X and Y.  (Can you spot the spam?  can
you spot the relevant documents?)

For spotting problems (X versus not-X), there are often more
appropriate success metrics than accuracy.  A very popular one from
information retrieval is the \concept{precision}/\concept{recall}
metric.  Precision asks the question: of all the X's that you found,
how many of them were actually X's?  Recall asks: of all the X's that
were out there, how many of them did you find?\sidenote{A colleague
  make the analogy to the US court system's saying ``Do you promise to
  tell the whole truth and nothing but the truth?''  In this case, the
  ``whole truth'' means high recall and ``nothing but the truth''
  means high precision.''}  Formally, precision and recall are defined
as:
\begin{align}
P &= \frac I S \\
R &= \frac I T \\
S &= \text{number of Xs that your system found} \\
T &= \text{number of Xs in the data} \\
I &= \text{number of correct Xs that your system found}
\end{align}
Here, $S$ is mnemonic for ``System,'' $T$ is mnemonic for ``Truth''
and $I$ is mnemonic for ``Intersection.''  It is generally accepted
that $0/0 = 1$ in these definitions.  Thus, if you system found
nothing, your precision is always perfect; and if there is nothing to
find, your recall is always perfect.

\TODOFigure{prac:spam}{show a bunch of emails spam/nospam sorted by model predicion, not perfect}

Once you can compute precision and recall, you are often able to
produce \concept{precision/recall curves}.  Suppose that you are
attempting to identify spam.  You run a learning algorithm to make
predictions on a test set.  But instead of just taking a ``yes/no''
answer, you allow your algorithm to produce its confidence.  For
instance, in perceptron, you might use the distance from the
hyperplane as a confidence measure.  You can then sort all of your
test emails according to this ranking.  You may put the most spam-like
emails at the top and the least spam-like emails at the bottom, like
in Figure~\ref{fig:prac:spam}.

\thinkaboutit{How would you get a confidence out of a decision tree or KNN?}

\TODOFigure{prac:prcurve}{precision recall curve}

Once you have this sorted list, you can choose how aggressively you
want your spam filter to be by setting a threshold \emph{anywhere} on
this list.  One would hope that if you set the threshold very high,
you are likely to have high precision (but low recall).  If you set
the threshold very low, you'll have high recall (but low precision).
By considering \emph{every possible} place you could put this
threshold, you can trace out a curve of precision/recall values, like
the one in Figure~\ref{fig:prac:prcurve}.  This allows us to ask the
question: for some fixed precision, what sort of recall can I get.
Obviously, the closer your curve is to the upper-right corner, the
better.  And when comparing learning algorithms A and B you can say
that A \concept{dominates} B if A's precision/recall curve is always
higher than B's.

\Table{prac:f}%
  {Table of f-measures when varying precision and recall values.}%
  {@{}r|c@{~~~~}c@{~~~~}c@{~~~~}c@{~~~~}c@{~~~~}c@{}}{
    & 0.0 & 0.2 & 0.4 & 0.6 & 0.8 & 1.0 \\
\hline
0.0& \showfscore{0.00} & \showfscore{0.00} & \showfscore{0.00} & \showfscore{0.00} & \showfscore{0.00} & \showfscore{0.00} \\
0.2& \showfscore{0.00} & \showfscore{0.20} & \showfscore{0.26} & \showfscore{0.30} & \showfscore{0.32} & \showfscore{0.33} \\
0.4& \showfscore{0.00} & \showfscore{0.26} & \showfscore{0.40} & \showfscore{0.48} & \showfscore{0.53} & \showfscore{0.57} \\
0.6& \showfscore{0.00} & \showfscore{0.30} & \showfscore{0.48} & \showfscore{0.60} & \showfscore{0.68} & \showfscore{0.74} \\
0.8& \showfscore{0.00} & \showfscore{0.32} & \showfscore{0.53} & \showfscore{0.68} & \showfscore{0.80} & \showfscore{0.88} \\
1.0& \showfscore{0.00} & \showfscore{0.33} & \showfscore{0.57} & \showfscore{0.74} & \showfscore{0.88} & \showfscore{1.00}
}

Precision/recall curves are nice because they allow us to visualize
many ways in which we could use the system.  However, sometimes we
like to have a \emph{single number} that informs us of the quality of
the solution.  A popular way of combining precision and recall into a
single number is by taking their harmonic mean.  This is known as the
balanced f-measure (or f-score):
\begin{equation} \label{eq:f}
F = \frac {2\times P \times R} {P + R}
\end{equation}
The reason that you want to use a harmonic mean rather than an
arithmetic mean (the one you're more used to) is that it favors
systems that achieve roughly equal precision and recall.  In the
extreme case where $P=R$, then $F=P=R$.  But in the imbalanced case,
for instance $P=0.1$ and $R=0.9$, the overall f-measure is a modest
$0.18$.  Table~\ref{tab:prac:f} shows f-measures as a function of
precision and recall, so that you can see how important it is to get
balanced values.

In some cases, you might believe that precision is more important than
recall.  This idea leads to the \emph{weighted} f-measure, which is
parameterized by a weight $\be \in [0,\infty)$ (beta):
\begin{equation} \label{eq:f-weighted}
F_\be = \frac {(1 + \be^2) \times P \times R} {\be^2 \times P + R}
\end{equation}
For $\be=1$, this reduces to the standard f-measure.  For $\be = 0$, it
focuses entirely on recall and for $\be \rightarrow \infty$ it focuses
entirely on precision.  The interpretation of the weight is that
$F_\be$ measures the performance for a user who cares $\be$ times as
much about precision as about recall.

One thing to keep in mind is that precision and recall (and hence
f-measure) depend crucially on which class is considered the thing you
wish to find.  In particular, if you take a binary data set if flip
what it means to be a positive or negative example, you will end up
with completely difference precision and recall values.  It is
\emph{not} the case that precision on the flipped task is equal to
recall on the original task (nor vice versa).  Consequently, f-measure
is also not the same.  For some tasks where people are less sure about
what they want, they will occasionally report two sets of
precision/recall/f-measure numbers, which vary based on which class is
considered the thing to spot.

There are other standard metrics that are used in different
communities.  For instance, the medical community is fond of the
\concept{sensitivity}/\concept{specificity} metric.  A sensitive
classifier is one which almost always finds everything it is looking
for: it has high recall.  In fact, sensitivity is exactly the same as
recall.  A specific classifier is one which does a good job \emph{not}
finding the things that it doesn't want to find.  Specificity is
precision on the negation of the task at hand.

You can compute curves for sensitivity and specificity much like those
for precision and recall.  The typical plot, referred to as the
\concept{receiver operating characteristic} (or \concept{ROC curve})
plots the sensitivity against $1 - \text{specificity}$.  Given an ROC
curve, you can compute the \concept{area under the curve} (or
\concept{AUC}) metric, which also provides a meaningful single number
for a system's performance.  Unlike f-measures, which tend to be low
because the require agreement, AUC scores tend to be very high, even
for not great systems.  This is because random chance will give you an
AUC of $0.5$ and the best possible AUC is $1.0$.

The main message for evaluation metrics is that you should choose
whichever one makes the most sense.  In many cases, several might make
sense.  In that case, you should do whatever is more commonly done in
your field.  There is no reason to be an outlier without cause.

\section{Cross Validation}

In Chapter~\ref{sec:dt}, you learned about using development data (or
held-out data) to set hyperparameters.  The main disadvantage to the
development data approach is that you throw out some of your training
data, just for estimating one or two hyperparameters.

\newalgorithm{prac:cv}%
  {\FUN{CrossValidate}(\VAR{LearningAlgorithm}, \VAR{Data}, \VAR{K})}
  {
\SETST{$\hat\ep$}{$\infty$}
  \COMMENT{store lowest error encountered so far}
\SETST{$\hat\al$}{unknown}
  \COMMENT{store the hyperparameter setting that yielded it}
\FORALL{hyperparameter settings $\al$}
\SETST{err}{\emptylist}
  \COMMENT{keep track of the $K$-many error estimates}
\FOR{\VAR{k} = \CON{1} \TO \VAR{K}}
\SETST{train}{$\{ (\VARm{\vx_n}, \VARm{y_n}) \in \VAR{Data} : \VARm{n}\mod \VARm{K} \neq k-1 \}$}
\SETST{test}{$\{ (\VARm{\vx_n}, \VARm{y_n}) \in \VAR{Data} : \VARm{n}\mod \VARm{K} = k-1 \}$}
\COMMENT{test every $K$th example}
\SETST{model}{Run \VAR{LearningAlgorithm} on \VAR{train}}
\SETST{err}{\VAR{err} \pushlist{error of \VAR{model} on \VAR{test}}}
\COMMENT{add current error to list of errors}
\ENDFOR
\SETST{avgErr}{mean of set \VAR{err}}
\IF{$\VARm{avgErr} < \VARm{\hat\ep}$}
\SETST{$\hat\ep$}{$\VARm{avgErr}$}
\COMMENT{remember these settings}
\SETST{$\hat\al$}{$\VARm{\al}$}
\COMMENT{because they're the best so far~}
\ENDIF
\ENDFOR
}

An alternative is the idea of \concept{cross validation}.  In cross
validation, you break your training data up into $10$ equally-sized
partitions.  You train a learning algorithm on $9$ of them and test it
on the remaining $1$.  You do this $10$ times, each time holding out a
different partition as the ``development'' part.  You can then average
your performance over all ten parts to get an estimate of how well
your model will perform in the future.  You can repeat this process
for every possible choice of hyperparameters to get an estimate of
which one performs best.  The general $K$-fold cross validation
technique is shown in Algorithm~\ref{alg:prac:cv}, where $K=10$ in the
preceeding discussion.

In fact, the development data approach can be seen as an approximation
to cross validation, wherein only one of the $K$ loops (line 5 in
Algorithm~\ref{alg:prac:cv}) is executed.

Typical choices for $K$ are $2$, $5$, $10$ and $N-1$.  By far the most
common is $K=10$: 10-fold cross validation.  Sometimes $5$ is used for
efficiency reasons.  And sometimes $2$ is used for subtle statistical
reasons, but that is quite rare.  In the case that $K=N-1$, this is
known as \concept{leave-one-out cross validation} (or abbreviated as
\concept{LOO cross validation}).  After running cross validation, you
have two choices.  You can either select one of the $K$ trained models
as your final model to make predictions with, or you can train a
\emph{new} model on all of the data, using the hyperparameters
selected by cross-validation.  If you have the time, the latter is
probably a better options.

\newalgorithm{alg:prac:loo}%
  {\FUN{KNN-Train-LOO}(\VAR{$\mat D$})}
  {
\SETST{err$_k$}{\CON{0}, $\forall 1 \leq \VARm{k} \leq \VARm{N}-\CON{1}$}
\COMMENT{err$_k$ stores how well you do with $k$NN}
\FOR{\VAR{$n$} $=\CON{1}$ \TO \VAR{$N$}}
\SETST{S$_m$}{$\langle\norm{\VARm{\vx_n} - \VARm{\vx_m}}, \VARm{m}\rangle$, $\forall \VARm{m} \neq \VARm{n}$}
  \COMMENT{compute distances to other points}
\SETST{S}{\FUN{sort}(\VAR{S})}
\COMMENT{put lowest-distance objects first}
\SETST{$\hat y$}{0}
\COMMENT{current label prediction}
\FOR{$\VARm{k} = \CON{1}$ \TO $\VAR{N}-\CON{1}$}
\SETST{$\langle$dist,m$\rangle$}{\VAR{S$_k$}}
\SETST{$\hat y$}{$\VARm{\hat y} + \VARm{y_m}$}
\COMMENT{let $k$th closest point vote}
\IF{$\VARm{\hat y} \neq \VARm{y_m}$}
\SETST{err$_k$}{\VAR{err$_k$} + \CON{1}}
\COMMENT{one more error for $k$NN}
\ENDIF
\ENDFOR
\ENDFOR
\RETURN $\argmin_k \VARm{err_k}$
\COMMENT{return the $K$ that achieved lowest error}
}

It may seem that LOO cross validation is prohibitively expensive to
run.  This is true for most learning algorithms \emph{except for}
$K$-nearest neighbors.  For KNN, leave-one-out is actually very
natural.  We loop through each training point and ask ourselves
whether this example would be correctly classified for all different
possible values of $K$.  This requires only as much computation as
computing the $K$ nearest neighbors for the highest value of $K$.
This is such a popular and effective approach for KNN classification
that it is spelled out in Algorithm~\ref{alg:prac:loo}.

Overall, the main advantage to cross validation over development data
is robustness.  The main advantage of development data is speed.

One warning to keep in mind is that the goal of both cross validation
and development data is to estimate how well you will do in the
future.  This is a question of statistics, and holds \emph{only if}
your test data really looks like your training data.  That is, it is
drawn from the same distribution.  In many practical cases, this is
not entirely true.

For example, in person identification, we might try to classify every
pixel in an image based on whether it contains a person or not.  If we
have $100$ training images, each with $10,000$ pixels, then we have a
total of $1m$ training examples.  The classification for a pixel in
image $5$ is highly dependent on the classification for a neighboring
pixel in the same image.  So if one of those pixels happens to fall in
training data, and the other in development (or cross validation)
data, your model will do unreasonably well.  In this case, it is
important that when you cross validate (or use development data), you
do so over \emph{images}, not over \emph{pixels}.  The same goes for
text problems where you sometimes want to classify things at a word
level, but are handed a collection of documents.  The important thing
to keep in mind is that it is the \emph{images} (or documents) that
are drawn independently from your data distribution and \emph{not} the
pixels (or words), which are drawn dependently.

\section{Hypothesis Testing and Statistical Significance}

\begin{vignette}{The Lady Drinking Tea}
  story
\end{vignette}

Suppose that you've presented a machine learning solution to your boss
that achieves $7\%$ error on cross validation.  Your nemesis, Gabe,
gives a solution to your boss that achieves $6.9\%$ error on cross
validation.  How impressed should your boss be?  It depends.  If this
$0.1\%$ improvement was measured over $1000$ examples, perhaps not too
impressed.  It would mean that Gabe got exactly one more example right
than you did.  (In fact, they probably got $15$ more right and $14$ more
wrong.)  If this $0.1\%$ impressed was measured over $1,000,000$
examples, perhaps this is more impressive.

This is one of the most fundamental questions in statistics.  You have
a scientific hypothesis of the form ``Gabe's algorithm is better than
mine.''  You wish to test whether this hypothesis is true.  You are
testing it against the \concept{null hypothesis}, which is that Gabe's
algorithm is no better than yours.  You've collected data (either
$1000$ or $1m$ data points) to measure the strength of this
hypothesis.  You want to ensure that the difference in performance of
these two algorithms is \concept{statistically significant}: i.e., is
probably not just due to random luck.  (A more common question
statisticians ask is whether one drug treatment is better than
another, where ``another'' is either a placebo or the competitor's
drug.)

There are about $\infty$-many ways of doing \concept{hypothesis
  testing}.  Like evaluation metrics and the number of folds of cross
validation, this is something that is very discipline specific.  Here,
we will discuss two popular tests: the \concept{paired t-test} and
\concept{bootstrapping}.  These tests, and other statistical tests,
have underlying assumptions (for instance, assumptions about the
distribution of observations) and strengths (for instance, small or
large samples).  In most cases, the goal of hypothesis testing is to
compute a \concept{p-value}: namely, the probability that the observed
difference in performance was by chance.  The standard way of
reporting results is to say something like ``there is a $95\%$ chance
that this difference was not by chance.''  The value $95\%$ is
arbitrary, and occasionally people use weaker ($90\%$) test or
stronger ($99.5\%$) tests.

The \concept{t-test} is an example of a \concept{parametric test}.  It
is applicable when the null hypothesis states that the difference
between two responses has mean zero and unknown variance.  The t-test
actually assumes that data is distributed according to a Gaussian
distribution, which is probably \emph{not} true of binary responses.
Fortunately, for large samples (at least a few hundred), binary
samples are well approximated by a Gaussian distribution.  So long as
your sample is sufficiently large, the t-test is reasonable either for
regression or classification problems.

\Table{prac:t}%
  {Table of significance values for the t-test.}%
  {c|c}{
$t$ & significance \\
\hline
$\geq 1.28$ & $90.0\%$ \\
$\geq 1.64$ & $95.0\%$ \\
$\geq 1.96$ & $97.5\%$ \\
$\geq 2.58$ & $99.5\%$
}

Suppose that you evaluate two algorithm on $N$-many examples.  On each
example, you can compute whether the algorithm made the correct
prediction.  Let $a_1, \dots, a_N$ denote the error of the first
algorithm on each example.  Let $b_1, \dots, b_N$ denote the error of
the second algorithm.  You can compute $\mu_a$ and $\mu_b$ as the
means of $\vec a$ and $\vec b$, respecitively.  Finally, center the
data as $\hat a = \vec a - \mu_a$ and $\hat b = \vec b - \mu_b$.
The t-statistic is defined as:
\begin{equation} \label{eq:prac:t}
t =
\left( \mu_a - \mu_b \right)
\sqrt{
 \frac {N (N-1)}
       {\sum_n ( \hat a_n - \hat b_n )^2}
}
\end{equation}
After computing the $t$-value, you can compare it to a list of values
for computing \concept{confidence intervals}.  Assuming you have a lot
of data ($N$ is a few hundred or more), then you can compare your
$t$-value to Table~\ref{tab:prac:t} to determine the significance
level of the difference.

\thinkaboutit{What does it mean for the means $\mu_a$ and $\mu_b$ to
  become further apart?  How does this affect the $t$-value?  What
  happens if the variance of $\vec a$ increases?}

One disadvantage to the t-test is that it cannot easily be applied to
evaluation metrics like f-score.  This is because f-score is a
computed over an entire test set and does not decompose into a set of
individual errors.  This means that the t-test cannot be applied.

Fortunately, \concept{cross validation} gives you a way around this
problem.  When you do $K$-fold cross validation, you are able to
compute $K$ error metrics over the same data.  For example, you might
run $5$-fold cross validation and compute f-score for every fold.
Perhaps the f-scores are $92.4$, $93.9$, $96.1$, $92.2$ and $94.4$.
This gives you an average f-score of $93.8$ over the $5$ folds.  The
standard deviation of this set of f-scores is:
\begin{align} \label{eq:prac:stddev}
\si
&= \sqrt{ \frac 1 {N-1} \sum_n (a_i - \mu)^2 } \\
&= \sqrt{ \frac 1 4 (1.96 + 0.01 + 5.29 + 2.56 + 0.36) } \\
&= 1.595
\end{align}
You can now assume that the distribution of scores is approximately
Gaussian.  If this is true, then approximately $70\%$ of the
probability mass lies in the range $[\mu-\si, \mu+\si]$; $95\%$ lies
in the range $[\mu-2\si,\mu+2\si]$; and $99.5\%$ lies in the range
$[\mu-3\si,\mu+3\si]$.  So, if we were comparing our algorithm against
one whose average f-score was $90.6\%$, we could be $95\%$ certain
that our superior performance was not due to chance.\sidenote{Had we
  run $10$-fold cross validation we might be been able to get tighter
  confidence intervals.}

\textcolor{darkred}{\bf WARNING:} A confidence of $95\%$ does not mean
``There is a $95\%$ chance that I am better.''  All it means is that
if I reran the same experiment $100$ times, then in $95$ of those
experiments I would still win.  These are \emph{very} different
statements.  If you say the first one, people who know about
statistics will get very mad at you!

One disadvantage to cross validation is that it is computationally
expensive.  More folds typically leads to better estimates, but every
new fold requires training a new classifier.  This can get very time
consuming.  The technique of \concept{bootstrapping} (and closely
related idea of \concept{jack-knifing} can address this problem.

Suppose that you didn't want to run cross validation.  All you have is
a single held-out test set with $1000$ data points in it.  You can run
your classifier and get predictions on these $1000$ data points.  You
would like to be able to compute a metric like f-score on this test
set, but also get confidence intervals.  The idea behind bootstrapping
is that this set of $1000$ is a random draw from some distribution.
We would like to get multiple random draws from this distribution on
which to evaluate.  We can \emph{simulate} multiple draws by
repeatedly subsampling from these $1000$ examples, with replacement.

To perform a \emph{single} bootstrap, you will sample $1000$ random
points from your test set of $1000$ random points.  This sampling must
be done with replacement (so that the same example can be sampled more
than once), otherwise you'll just end up with your original test set.
This gives you a bootstrapped sample.  On this sample, you can compute
f-score (or whatever metric you want).  You then do this $99$ more
times, to get a $100$-fold bootstrap.  For each bootstrapped sample,
you will be a different f-score.  The mean and standard deviation of
this set of f-scores can be used to estimate a confidence interval for
your algorithm.

\newalgorithm{prac:bootstrap}%
  {\FUN{BootstrapEvaluate}(\VAR{$\vec y$}, \VAR{$\hat{\vec y}$}, \VAR{NumFolds})}
  {
\SETST{scores}{\emptylist}
\FOR{\VAR{k} = \CON{1} \TO \VAR{NumFolds}}
\SETST{truth}{\emptylist}
\COMMENT{list of values we want to predict}
\SETST{pred}{\emptylist}
\COMMENT{list of values we actually predicted}
\FOR{\VAR{n} = \CON{1} \TO \VAR{N}}
\SETST{m}{uniform random value from \CON{1} to \VAR{N}}
\COMMENT{sample a test point}
\SETST{truth}{\VAR{truth} \pushlist \VAR{$y_m$}}
\COMMENT{add on the truth}
\SETST{pred}{\VAR{pred} \pushlist \VAR{$\hat y_m$}}
\COMMENT{add on our prediction}
\ENDFOR
\SETST{scores}{\VAR{scores} \pushlist \FUN{f-score}(\VAR{truth}, \VAR{pred})}
\COMMENT{evaluate}
\ENDFOR
\RETURN (\FUN{mean}(\VAR{scores}), \FUN{stddev}(\VAR{scores}))
}

The bootstrap resampling procedure is sketched in
Algorithm~\ref{alg:prac:bootstrap}.  This takes three arguments: the
true labels $\vec y$, the predicted labels $\hat{\vec y}$ and the
number of folds to run.  It returns the mean and standard deviation
from which you can compute a confidence interval.

\section{Debugging Learning Algorithms}

Learning algorithms are notoriously hard to debug, as you may have
already experienced if you have implemented any of the models
presented so far.  The main issue is that when a learning algorithm
doesn't learn, it's unclear if this is because there's a bug or
because the learning problem is too hard (or there's too much noise,
or \dots).  Moreover, sometimes bugs lead to learning algorithms
performing \emph{better} than they should: these are especially hard
to catch (and always a bit disappointing when you do catch them).

Obviously if you have a reference implementation, you can attempt to
match its output.  Otherwise, there are two things you can do to try
to help debug.  The first is to do everything in your power to get the
learning algorithm to overfit.  If it cannot \emph{at least} overfit
the training data, there's definitely something wrong.  The second is
to feed it some tiny hand-specified two dimensional data set on which
you \emph{know} what it should do and you can plot the output.

The easiest way to try to get a learning algorithm to overfit is to
add a new feature to it.  You can call this feature the
\feat{CheatingIsFun} feature\footnote{Note: cheating is actually
  \emph{not} fun and you shouldn't do it!}.  The feature value
associated with this feature is $+1$ if this is a positive example and
$-1$ (or zero) if this is a negative example.  In other words, this
feature is a \emph{perfect} indicator of the class of this example.

If you add the \feat{CheatingIsFun} feature and your algorithm does
not get near $0\%$ training error, this could be because there are too
many noisy features confusing it.  You could either remove a lot of
the other features, or make the feature value for \feat{CheatingIsFun}
either $+100$ or $-100$ so that the algorithm \emph{really} looks at
it.  If you do this and your algorithm still cannot overfit then you
likely have a bug.  (Remember to remove the \feat{CheatingIsFun}
feature from your final implementation!)

A second thing to try is to hand-craft a data set on which you know
your algorithm should work.  This is also useful if you've managed to
get your model to overfit and have simply noticed that it does not
generalize.  For instance, you could run KNN on the XOR data.  Or you
could run perceptron on some easily linearly separable data (for
instance positive points along the line $x_2 = x_1 + 1$ and negative
points along the line $x_2 = x_1 - 1$).  Or a decision tree on nice
axis-aligned data.

When debugging on hand-crafted data, remember whatever you know about
the models you are considering.  For instance, you know that the
perceptron should converge on linearly separable data, so try it on a
linearly separable data set.  You know that decision trees should do
well on data with only a few relevant features, so make your label
some easy combination of features, such as $y = x_1 \lor (x_2 \land
\lnot x_3)$.  You know that KNN should work well on data sets where
the classes are well separated, so try such data sets.

The most important thing to keep in mind is that a \emph{lot} goes in
to getting good test set performance.  First, the model has to be
right for the data.  So crafting your own data is helpful.  Second,
the model has to fit the training data well, so try to get it to
overfit.  Third, the model has to generalize, so make sure you tune
hyperparameters well.

\TODOFigure{prac:imageanswers}{object recognition answers}

\begin{comment}
\section{Human Knowledge is Everywhere}

\TODOFigure{prac:boc}{BOC repr of one positive and one negative review}

Alternatively, one can use bags of characters, a representation much
like bag of pixels.  As you might expect, this is a terrible
representation for most text categorization problems.  However,
instead of using a ``bag of X'' representation, one can you a ``bag of
(sequences of X)'' representation.  For example, we could have one
feature for any pair of adjacent words\sidenote{This might help
  recognize that ``real estate'' is different than ``real'' +
  ``estate''.} or adjacent characters.  As these sequences grow
longer, the size of the feature space grows, but we begin to see more
meaningful units.  Figure~\ref{fig:prac:boseq} shows \emph{test
  performance} for both bag of words and bag of characters, for
different sequence lengths.  As you can see, once the character
sequence length hits $5$, it performs roughly equally to bag of
words (this is because most words are about $4$ characters long, plus
a space).  As the sequence lengths grow, the models begin to overfit
and performance drops.

One can take this idea further.  Instead of sequences of characters,
we can consider sequences of \emph{bits}.  

\thinkaboutit{Can you think of a text categorization problem where bag
  of characters might actually be a reasonable representation?}

One issue in language is that words are often not the right level of
granularity for classification.  For instance, ``real estate'' is very
different than ``real'' + ``estate'', so in one sense a word is too
small of a unit.  On the other hand, ``fortunate'' and ``fortunately''
are really not very different, so a word can also be too large of a
unit.  A possible way to address this is to either move ``up'' and
consider \emph{sequences of words}, in addition to single words.
Another possibility is to move ``down'' and consider \emph{sequences
  of characters}.

\TODOFigure{prac:boc}{BOC repr of one positive and one negative review
  (negative first)}

As an extreme example in text, one can use a bag of characters
representation.  This is more analogous to the pixel representation in
images.  Figure~\ref{fig:prac:boc} shows such a representation for a
difference positive and negative document pair.  In this case it is
near impossible to tell the difference!  (It turns out that the first
one is negative.)



gzip example

\end{comment}

TODO: answers to image questions


%%% Local Variables: 
%%% mode: latex
%%% TeX-master: "courseml"
%%% End: 
