%%% Local Variables: 
%%% mode: latex
%%% TeX-master: "courseml"
%%% End: 

\chapter{About this Book} \label{sec:intro}

\newthought{Machine learning is a broad and fascinating field.} 
Even today, machine learning technology runs a substantial part of your life, often without you knowing it.
Any plausible approach to artificial intelligence must involve learning, at some level, if for no other reason than it's hard to call a system intelligent if it \emph{cannot} learn.
Machine learning is also fascinating in its own right for the philosophical questions it raises about what it means to learn and succeed at tasks.

Machine learning is also a very broad field, and attempting to cover everything would be a pedagogical disaster.
It is also so quickly moving that any book that attempts to cover the latest developments will be outdated before it gets online.
Thus, this book has two goals.
First, to be a gentle introduction to what is a very deep field.
Second, to provide readers with the skills necessary to pick up new technology as it is developed.

\section{How to Use this Book}

This book is designed to be read linearly, since it's goal is not to be a generic reference.
That said, once you get through chapter 6, you can pretty much jump anywhere.
When I teach a one-semester undergraduate course, I typically cover the chapter 1-13, sometimes skipping 7 or 9 or 10 depending on time and interest.
For a graduate course for students with no prior machine learning background, I would very quickly blaze through 1-4, then cover the rest, augmented with some additional reading.
At the end of some of the chapters, I have provided some links to ``advanced papers'' on that chapter's topic, which I often use for extra credit for undergraduate courses and required reading for graduates.


\section{Why Another Textbook?}

The purpose of this book is to provide a \emph{gentle} and
\emph{pedagogically organized} introduction to the field.  This is in
contrast to most existing machine learning texts, which tend to
organize things topically, rather than pedagogically (an exception is
Mitchell's book\mycite{mitchell97book}, but unfortunately that is
getting more and more outdated).  This makes sense for researchers in
the field, but less sense for learners.  A second goal of this book is
to provide a view of machine learning that focuses on ideas and
models, not on math.  It is not possible (or even advisable) to avoid
math.  But math should be there to \emph{aid} understanding, not
hinder it.  Finally, this book attempts to have minimal dependencies,
so that one can fairly easily pick and choose chapters to read.  When
dependencies exist, they are listed at the start of the chapter.

The \emph{audience} of this book is anyone who knows differential
calculus and discrete math, and can program reasonably well.  (A
little bit of linear algebra and probability will not hurt.)  An
undergraduate in their fourth or fifth semester should be fully
capable of understanding this material.  However, it should also be
suitable for first year graduate students, perhaps at a slightly
faster pace.

\section{Organization and Auxilary Material}

There is an associated web page, \bookurl, which contains an online
copy of this book, as well as associated code and data.  It also
contains errata. Please submit bug reports on github: \url{github.com/hal3/ciml}.

\section{Acknowledgements}

{\bf Acknowledgements:} I am indebted to many people for this book.
My teachers, especially Rami Grossberg (from whom the title of this
book was borrowed) and Stefan Schaal.  Students who have taken machine
learning from me over the past ten years, including those who suffered
through the initial versions of the class before I figured out how to
teach it.  Especially Scott Alfeld, Josh de Bever, Cecily Heiner,
Jeffrey Ferraro, Seth Juarez, John Moeller, JT Olds, Piyush Rai.
People who have helped me edit, and who have submitted bug reports,
including \TODO, but also check github for the latest list of
contributors!

\begin{comment}

Why written:
- undergrads
- focus on ideas, not math
- pedagogy
- minimal dependencies
- gentle

For whom:
- people who know nothing about ml
- know differential calculus
- know discrete math
- can write simple programs
- could be taken by sophomores or juniors!
- or first year grads who want a gentle intro

Online stuff:
- book
- data
- code
- solutions
- wiki

Course plans:
- dependencies
- one semester undergrad
- one semester grad

Acknowledgements:
- teachers
- students
- editors


\end{comment}
